% 用 xelatex 编译
%\documentclass[a4paper,cs4size,hyperref,fancyhdr]{ctexbook}
\documentclass[UTF8,openright]{ctexbook}
\usepackage{ecnu}

%%%%%%%%%%%%%%%%%%%%%%%%%%%%%%%%%%%%%%%%%%%%%%%%%%%%%%%%%%%%%%%%%%%%%
\begin{document}
\zihao{-4}

%%%%% ===== 封面 =====
%%%%% ===== 中文封面信息
\graduateyear{2019} % 毕业年份
\class{} % 分类号(数值线性代数是 O241.6)
\ctitle{我的论文\\标题}
\def\cctitle{我的论文标题} % 在原创性声明中使用, 不能出现手工换行
\caffil{计算机科学技术系}
\cmajor{计算机应用技术} % 计算数学
\cdirection{自然语言处理} % 数值代数
\csupervisorA{吴苑斌\, 副教授}
% \csupervisorB{吴苑斌\, 副教授}
% \csupervisor{吴苑斌\, 副教授}
\cauthor{孙长志}
%\csupervisorA{$\ast\ast\ast$}
%\csupervisorB{$\ast\ast\ast$}
%\cauthor{$\ast\ast\ast$}
\def\ccauthor{\@cauthor} % 在答辩委员会中使用
\studentid{52164500005}
\cdate{2019 年 06 月}

%%%%% ===== 英文封面信息
\etitle{This is my \\ Thesis Title}
\eaffil{Computer Science and Technology} 
\emajor{Computer Application Technology} % Computational Mathematics
\edirection{Natural Language Processing} % Numerical Algebra
% \edirectionB{Automated Reasoning} 
\esupervisorA{Associate Professor Yuanbin Wu}
% \esupervisorB{Associate Professor Wu Yuan-Bin}
% \esupervisor{Associate Professor Wu Yuan-Bin}
\eauthor{Changzhi Sun}
%\esupervisorA{$\ast\ast\ast$}
%\esupervisorB{$\ast\ast\ast$}
%\eauthor{$\ast\ast\ast$}
\edate{06, 2019}

%%%%% ===== 生成封面 =====
\newgeometry{top=2.0cm,bottom=2.0cm,left=2.5cm,right=2.5cm}
{
\renewcommand{\baselinestretch}{1.6}
\makecover
}


%%%%% ===== 原创性声明与著作权使用声明 =====
%%%%% ===== 原创性声明与著作权使用声明
\newpage
\thispagestyle{empty}
\pdfbookmark[0]{原创性声明}{Declaration}

\vspace*{1em}

{
\linespread{1.4}\zihao{-4}
\centerline{\zihao{3}\STSong 华东师范大学学位论文原创性声明}

\bigskip

郑重声明:本人呈交的学位论文《\cctitle》,
是在华东师范大学攻读硕士/博士(请勾选)
学位期间,在导师的指导下进行的研究工作及取得的研究成果。
除文中已经注明引用的内容外,本论文不包含其他个人已经发表或撰写过的研究成果。
对本文的研究做出重要贡献的个人和集体,均已在文中作了明确说明并表示谢意。

\vspace{1em}


{\STSong 作者签名}:$\underline{\hspace{4cm}}$ \hfill
{\STSong 日\quad 期}: \qquad\ 年 \quad\ 月 \quad\ 日 \qquad\mbox{}

\vspace{4em}

\centerline{\zihao{3}\STSong 华东师范大学学位论文著作权使用声明}
\bigskip


《\cctitle》
系本人在华东师范大学攻读学位期间在导师指导下完成的硕士/博士(请勾选)学位论文,
本论文的著作权归本人所有。
本人同意华东师范大学根据相关规定保留和使用此学位论文,
并向主管部门和学校指定的相关机构送交学位论文的印刷版和电子版;
允许学位论文进入华东师范大学图书馆及数据库被查阅、借阅;
同意学校将学位论文加入全国博士、硕士学位论文共建单位数据库进行检索,
将学位论文的标题和摘要汇编出版,采用影印、缩印或者其它方式合理复制学位论文。

本学位论文属于(请勾选)
\newcounter{muni}
\begin{list}{{\hfill\upshape (\qquad)\ \arabic{muni}. }}{%
     \usecounter{muni}\leftmargin6.5em\labelwidth4.2em\labelsep0.2em
     \itemsep0.5em\itemindent0pt\parsep0pt\topsep0pt}

\item 经华东师范大学相关部门审查核定的“内部”或“涉密”学位论文*,
  于 \qquad 年 \quad 月 \quad  日解密, 解密后适用上述授权。

\item 不保密,适用上述授权。
\end{list}

\vskip0.8cm


{\STSong 导师签名}:$\underline{\hspace{4cm}}$ \hfill
{\STSong 本人签名}:$\underline{\hspace{4cm}}$

\bigskip

{\mbox{}\hfill 年\qquad 月\qquad  日 }

\vfill

\parbox[t]{0.946\textwidth}{\zihao{5}
*“涉密”学位论文应是已经华东师范大学学位评定委员会办公室或保密委员会
审定过的学位论文(需附获批的《华东师范大学研究生申请学位论文“涉密”审批表》
方为有效),未经上述部门审定的学位论文均为公开学位论文。
此声明栏不填写的,默认为公开学位论文,均适用上述授权)。\\
}}


%%%%% ===== 答辩委员会成员 =====
%%%% ===== 答辩委员会成员
\newpage
\thispagestyle{empty}
\pdfbookmark[0]{答辩委员会}{Committee}
\vspace*{2em}

\begin{center}\STSong\zihao{3}
%  \underline{\ 我的名字\ } 博士学位论文答辩委员会成员名单
 \underline{\ 我的名字\ } 硕士学位论文答辩委员会成员名单
\end{center}

\begin{center}\zihao{4}
\renewcommand{\arraystretch}{1.4}
  \begin{tabular}{|c|c|c|c|} \hline
   ~~~~~姓~名~~~~~ & ~~~~~职~称~~~~~ 
   & \hspace{6em}单~位\hspace{6em} & ~~~备~注~~~\\\hline
         XXX    &   教授    &  XXXXX大学XXX系  & 主席  \\ \hline
         XXX    &   教授    &  XXXXX大学XXX系  &       \\ \hline
         XXX    &   教授    &  XXXXX大学XXX系  &       \\ \hline
                &           &                   &       \\ \hline
                &           &                   &       \\ \hline
  \end{tabular}
\end{center}


\frontmatter
%\restoregeometry
%%%%% ===== 中文摘要 =====
\pagestyle{plain}
\newgeometry{top=3.0cm,bottom=2.7cm,left=2.5cm,right=2.5cm}
{
%%% 中文摘要
\clearpage%{\pagestyle{empty}\cleardoublepage}
\thispagestyle{plain}
\phantomsection
\addcontentsline{toc}{chapter}{摘\quad 要}

\centerline{\zihao{-3}\heiti 摘\quad 要}

\linespread{1.4}\zihao{-4} \bigskip

这里是中文摘要, 这里是中文摘要, 这里是中文摘要, 这里是中文摘要,
这里是中文摘要, 这里是中文摘要, 这里是中文摘要, 这里是中文摘要,
这里是中文摘要, 这里是中文摘要, 这里是中文摘要, 这里是中文摘要.

这里是中文摘要, 这里是中文摘要, 这里是中文摘要, 这里是中文摘要,
这里是中文摘要, 这里是中文摘要, 这里是中文摘要, 这里是中文摘要,
这里是中文摘要, 这里是中文摘要, 这里是中文摘要, 这里是中文摘要.

\bigskip

\noindent{\zihao{4}\heiti 关键词:}
关键词, 关键词, 关键词

\clearpage{\pagestyle{empty}\cleardoublepage}

%%%%% ===== 英文摘要 =====
% Abstract
\clearpage%{\pagestyle{empty}\cleardoublepage}
\thispagestyle{plain}
\phantomsection
\addcontentsline{toc}{chapter}{Abstract}

\centerline{\zihao{3}\bfseries ABSTRACT}

\linespread{1.4}\zihao{-4}
\bigskip


Abstract in English. Abstract in English. Abstract in English.
Abstract in English. Abstract in English. Abstract in English.
Abstract in English. Abstract in English. Abstract in English.
Abstract in English. Abstract in English. Abstract in English.

Abstract in English. Abstract in English. Abstract in English.
Abstract in English. Abstract in English. Abstract in English.
Abstract in English. Abstract in English. Abstract in English.
Abstract in English. Abstract in English. Abstract in English.


\bigskip
\noindent\textbf{\zihao{4} Keywords:}
key words, key words, key words





\clearpage{\pagestyle{empty}\cleardoublepage}
}
\pagestyle{plain}

%%%%% ===== 生成目录

\newgeometry{top=3.0cm,bottom=3.0cm,left=2.5cm,right=2.5cm}
{
\setcounter{tocdepth}{1}
\phantomsection
\addcontentsline{toc}{chapter}{目录}
\tableofcontents

\newpage
\listoffigures % 插图目录

\listoftables  % 表格目录
}
\clearpage{\pagestyle{empty}\cleardoublepage}

%%%%%% ===== 正文部分 ===== %%%%%
\mainmatter
\newgeometry{top=2.5cm,bottom=2.5cm,left=3cm,right=3cm}
\linespread{1.4}\selectfont
%\setlength{\baselineskip}{0.88175cm}


\pagestyle{fancy}
\fancyhf{}  % 清除以前对页眉页脚的设置
\newcommand{\makeheadrule}{%% 定义页眉与正文间双隔线
  \makebox[0pt][l]{\rule[.7\baselineskip]{\headwidth}{0.3pt}}%0.4
  \rule[0.85\baselineskip]{\headwidth}{1.0pt}\vskip-.8\baselineskip}
\makeatletter
\renewcommand{\headrule}{%
  {\if@fancyplain\let\headrulewidth\plainheadrulewidth\fi\makeheadrule}}
\makeatother
\renewcommand{\chaptermark}[1]{\markboth{\CTEXthechapter \ #1}{}}
\renewcommand{\sectionmark}[1]{\markright{\thesection \ #1}{}}
%\fancyhead[RO,LE]{{\small\songti\rightmark}}     % 节标题
\fancyhead[CO,CE]{{\small\songti\leftmark}}      % 章标题
%\fancyhead[CO,CE]{华东师范大学博士学位论文}
%\fancyhead[RO,LE]{$\cdot$ {\small\thepage} $\cdot$}
\fancyfoot[CO,CE]{{\thepage}}


\chapter{绪论}
  


\chapter{基础准备}
  


\chapter{总结与展望}
  




\clearpage{\pagestyle{empty}\cleardoublepage}
\backmatter

%\fancyhf{}  % 清除以前对页眉页脚的设置
\fancyhead[CO,CE]{参考文献}
%\fancyhead[CO,LE]{$\cdot$ {\small\thepage} $\cdot$}
\fancyfoot[CO,CE]{{\thepage}}
\backmatter%使用此命令后,\addcontentsline{toc}{chapter}{参考文献}命令中页码与参考文献页码所在页码
\chapter*{ }
\setlength{\bibsep}{0.5ex} %设置参考文献行间距命令
\linespread{1.2}\selectfont
\addcontentsline{toc}{chapter}{参考文献}
\bibliographystyle{plainnat}
\bibliography{references}
%\pagestyle{fancy}

\clearpage{\pagestyle{empty}\cleardoublepage}
\chapter*{致谢}
\addcontentsline{toc}{chapter}{致谢}

这是致谢。
这是致谢。
这是致谢。
这是致谢。
这是致谢。
这是致谢。
这是致谢。

This template is heavily based on \citep{ecnu-thesis}. 
Thanks Chengchao Huang for his modifications.

\clearpage{\pagestyle{empty}\cleardoublepage}
\chapter*{在读期间发表的学术论文情况}
\addcontentsline{toc}{chapter}{在读期间发表的学术论文情况}
\begin{enumerate}[ {[}1{]} ]
    \item \textbf{Changzhi Sun} , Yeyun Gong, Yuanbin Wu, Ming Gong, Daxing Jiang, Man Lan, Shiliang Sun, and Nan Duan.
    Joint Type Inference on Entities and Relations via Graph Convolutional Networks.
    ACL 2019. (\textbf{CCF A 会议})
\item \textbf{Changzhi Sun} and Yuanbin Wu.
    Distantly Supervised Entity Relation Extraction with Adapted Manual Annotations.
    AAAI 2019. (\textbf{CCF A 会议})
\item  \textbf{Changzhi Sun} , Yuanbin Wu, Man Lan, Shiliang Sun, Wenting Wang, Kuang-Chih Lee, and Kewen Wu.
    Extracting Entities and Relations with Joint Minimum Risk Training.
    EMNLP 2018. (\textbf{CCF B 会议})
\item \textbf{Changzhi Sun}, Yuanbin Wu, Man Lan, Shiliang Sun, and Qi Zhang.
    Large-scale Opinion Relation Extraction with Distantly Supervised Neural Network.
    EACL 2017.
\end{enumerate}

\end{document}
