% 用 xelatex 编译
%\documentclass[a4paper,cs4size,hyperref,fancyhdr]{ctexbook}
\documentclass[UTF8,openright]{ctexbook}
\usepackage{ecnu}

%%%%%%%%%%%%%%%%%%%%%%%%%%%%%%%%%%%%%%%%%%%%%%%%%%%%%%%%%%%%%%%%%%%%%
\begin{document}

%%%%% ===== 封面 =====
%%%%% ===== 中文封面信息
\graduateyear{2019} % 毕业年份
\class{O156.2} % 分类号(数值线性代数是 O241.6)
\ctitle{我的论文\\标题}
\def\cctitle{我的论文标题} % 在原创性声明中使用, 不能出现手工换行
\caffil{计算机科学技术系}
\cmajor{计算机应用技术} % 计算数学
\cdirection{自然语言处理} % 数值代数
\csupervisorA{吴苑斌\, 副教授}
% \csupervisorB{吴苑斌\, 副教授}
% \csupervisor{吴苑斌\, 副教授}
\cauthor{我的名字}
%\csupervisorA{$\ast\ast\ast$}
%\csupervisorB{$\ast\ast\ast$}
%\cauthor{$\ast\ast\ast$}
\def\ccauthor{\@cauthor} % 在答辩委员会中使用
\studentid{52*********}
\cdate{2019 年 06 月}

%%%%% ===== 英文封面信息
\etitle{This is my \\ Thesis Title}
\eaffil{Computer Science and Technology} 
\emajor{Computer Application Technology} % Computational Mathematics
\edirection{Natural Language Processing} % Numerical Algebra
% \edirectionB{Automated Reasoning} 
\esupervisorA{Associate Professor Wu Yuan-Bin}
% \esupervisorB{Associate Professor Wu Yuan-Bin}
% \esupervisor{Associate Professor Wu Yuan-Bin}
\eauthor{My Name}
%\esupervisorA{$\ast\ast\ast$}
%\esupervisorB{$\ast\ast\ast$}
%\eauthor{$\ast\ast\ast$}
\edate{06, 2019}

%%%%% ===== 生成封面 =====
\newgeometry{top=2.0cm,bottom=2.0cm,left=2.5cm,right=2.5cm}
{
\renewcommand{\baselinestretch}{1.6}
\makecover
}


%%%%% ===== 原创性声明与著作权使用声明 =====
\include{preface/declaration}

%%%%% ===== 答辩委员会成员 =====
\include{preface/committee}

\frontmatter
%\restoregeometry
%%%%% ===== 中文摘要 =====
\pagestyle{plain}
\newgeometry{top=3.0cm,bottom=2.7cm,left=2.5cm,right=2.5cm}
{
\include{abstract/abstract_chs}
\clearpage{\pagestyle{empty}\cleardoublepage}

%%%%% ===== 英文摘要 =====
\include{abstract/abstract_eng}
\clearpage{\pagestyle{empty}\cleardoublepage}
}
\pagestyle{plain}

%%%%% ===== 生成目录

\newgeometry{top=3.0cm,bottom=3.0cm,left=2.5cm,right=2.5cm}
{
\setcounter{tocdepth}{1}
\phantomsection
\addcontentsline{toc}{chapter}{目录}
\tableofcontents

\newpage
\listoffigures % 插图目录

\listoftables  % 表格目录
}
\clearpage{\pagestyle{empty}\cleardoublepage}

%%%%%% ===== 正文部分 ===== %%%%%
\mainmatter
\newgeometry{top=2.5cm,bottom=2.5cm,left=3cm,right=3cm}
\linespread{1.4}\selectfont
%\setlength{\baselineskip}{0.88175cm}


\pagestyle{fancy}
\fancyhf{}  % 清除以前对页眉页脚的设置
\newcommand{\makeheadrule}{%% 定义页眉与正文间双隔线
  \makebox[0pt][l]{\rule[.7\baselineskip]{\headwidth}{0.3pt}}%0.4
  \rule[0.85\baselineskip]{\headwidth}{1.0pt}\vskip-.8\baselineskip}
\makeatletter
\renewcommand{\headrule}{%
  {\if@fancyplain\let\headrulewidth\plainheadrulewidth\fi\makeheadrule}}
\makeatother
\renewcommand{\chaptermark}[1]{\markboth{\CTEXthechapter \ #1}{}}
\renewcommand{\sectionmark}[1]{\markright{\thesection \ #1}{}}
%\fancyhead[RO,LE]{{\small\songti\rightmark}}     % 节标题
\fancyhead[CO,CE]{{\small\songti\leftmark}}      % 章标题
%\fancyhead[CO,CE]{华东师范大学博士学位论文}
%\fancyhead[RO,LE]{$\cdot$ {\small\thepage} $\cdot$}
\fancyfoot[CO,CE]{{\thepage}}


\chapter{绪论}
  这是绪论。
这是绪论。
这是绪论。
这是绪论。
这是绪论。
这是绪论。
这是绪论。
这是绪论。
这是绪论。
这是绪论。
这是绪论。
这是绪论。


\chapter{基础准备}
  这是基础准备。
这是基础准备。
这是基础准备。
这是基础准备。
这是基础准备。
这是基础准备。
这是基础准备。
这是基础准备。
这是基础准备。
这是基础准备。
这是基础准备。
这是基础准备。


\chapter{总结与展望}
  这是总结于展望。
这是总结于展望。
这是总结于展望。
这是总结于展望。
这是总结于展望。
这是总结于展望。
这是总结于展望。
这是总结于展望。
这是总结于展望。
这是总结于展望。
这是总结于展望。
这是总结于展望。
这是总结于展望。
这是总结于展望。
这是总结于展望。
这是总结于展望。
这是总结于展望。




\clearpage{\pagestyle{empty}\cleardoublepage}
\backmatter

%\fancyhf{}  % 清除以前对页眉页脚的设置
\fancyhead[CO,CE]{参考文献}
%\fancyhead[CO,LE]{$\cdot$ {\small\thepage} $\cdot$}
\fancyfoot[CO,CE]{{\thepage}}
\backmatter%使用此命令后,\addcontentsline{toc}{chapter}{参考文献}命令中页码与参考文献页码所在页码
\chapter*{ }
\setlength{\bibsep}{0.5ex} %设置参考文献行间距命令
\linespread{1.2}\selectfont
\addcontentsline{toc}{chapter}{参考文献}
\bibliographystyle{plainnat}
\bibliography{references}
%\pagestyle{fancy}

\clearpage{\pagestyle{empty}\cleardoublepage}
\chapter*{致谢}
\addcontentsline{toc}{chapter}{致谢}

这是致谢。
这是致谢。
这是致谢。
这是致谢。
这是致谢。
这是致谢。
这是致谢。
这是致谢。

\clearpage{\pagestyle{empty}\cleardoublepage}
\chapter*{在读期间发表的学术论文情况}
\addcontentsline{toc}{chapter}{在读期间发表的学术论文情况}
% \input{sections/papers.tex}



%\clearpage{\pagestyle{empty}\cleardoublepage}
%\chapter*{参与科研项目}
%\addcontentsline{toc}{chapter}{参与科研项目}

\newpage
\pagestyle{plain}
\fancyhf{}
% \input{sections/projects.tex}

\end{document}
